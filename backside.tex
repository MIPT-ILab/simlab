\thispagestyle{empty}
\begingroup
\small
\begin{flushleft}
УДК 004.42
\end{flushleft}

\begin{center}
\begin{normalsize}
\addfontfeature{LetterSpace=1.25}
\textsf{Рецензент}
\end{normalsize}

\medskip

Доктор технических наук, профессор кафедры информационных систем \\
управления и информационных технологий Московского государственного \\
университета приборостроения и информатики \textit{Т.~Ю.~Морозова}

\medskip

Кандидат технических наук, с.н.с., \\
начальник 4 отдела ОАО <<ИТМиВТ РАН>> \textit{Н.~Б.~Преображенский}

\end {center}

\textbf{Лабораторный практикум по программному моделированию}: учебно-методическое пособие / сост.: Е.~А.~Юлюгин, Г.~С.~Речистов, Н.~Н.~Щелкунов, Д.~А.~Гаврилов. --- М.: МФТИ, 2013. --- \pageref{page:lastpage}~с.\\

\medskip

Рассмотрены вопросы моделирования аппаратных средств однопроцессорных и многопроцессорных вычислительных систем. Основное внимание уделено технологиям построения программных симуляторов, в том числе на основе интерпретации, двоичной трансляции, аппаратной виртуализации, многопоточного исполнения и моделирования с использованием трасс. Учебное пособие подготовлено в «Лаборатории суперкомпьютерных технологий для биомедицины, фармакологии и малоразмерных структур» факультета радиотехники и кибернетики.

\medskip

Предназначено для студентов вузов, аспирантов и специалистов по средствам проектирования цифровых систем, разработке и исследованию проблемно-ориентированных архитектур ЭВМ и смежных направлений.

\vfill

\begin{center}
% The table in the bottom of the page.
\begin{tabular}{lp{0.6\textwidth}}
\hspace{4cm}                & \textcopyright~~Федеральное государственное автономное \\
                            & образовательное учреждение \\
                            & высшего профессионального образования \\
                            & «Московский физико-технический институт \\
                            & (государственный университет)», 2013
\end{tabular}

\end{center}

\endgroup
