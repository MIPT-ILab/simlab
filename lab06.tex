\chapter{Моделирование процессора архитектуры OpenRISC 1000}\label{chap:lab06}

Данная индивидуальная работа посвящена реализации модели компьютерной платформы, основанной на спецификации OpenRISC 10000~\cite{or1k-spec}. Модель строится на основе API Simics и оформляется как набор модулей и сценариев для данного симулятора.

\section{Спецификации OpenRISC 1000}

OpenRISC 1000 --- \todo 

Принцип организации спецификации --- это опциональность многих частей функциональности, что позволяет ограничиваться только важными минимально необходимыми для конкретной реализации частями.

\subsection{Набор регистров}

\todo Нарисовать рис.~\ref{fig:or1k-regs}

\begin{figure}[htbp]
\centering
\begin{tikzpicture}[>=latex, node distance=0.5cm, font=\small]
\begin{scope}[minimum width=4cm]
    \node[draw] (cpu) {\textsc{cpu}: chip16};
    \node[draw, below = of cpu.south west, anchor=north west] (graph) {\textsc{graphics}: graph16};
    \node[draw, below = of graph.south west, anchor=north west] (sound) {\textsc{sound}: snd16};
    \node[draw, below = of sound.south west, anchor=north west] (joystick) {\textsc{joystick}: input16};
\end{scope}
    
\begin{scope}[minimum width = 3.9cm, inner xsep = 0pt]
     \node[draw, rotate=90, right = 1cm of joystick.south east, anchor=north west] (memory-space) {\textsc{cspace}: memory-space};
    \node[draw, rotate=90, right = 1cm of memory-space.south, anchor = north] (ram) {\textsc{ram}: ram};
    
    \node[draw, rotate=90, left = 1cm of cpu.north west, anchor=south east] (video-space) {\textsc{vspace}: memory-space};
    \node[draw, rotate=90, left = 1cm of video-space.south, anchor = south] (video-ram) {\textsc{video-ram}: ram};
\end{scope}

\draw[<->] (cpu.east) -- (cpu.east -| memory-space.north);
\draw[<-] (graph.east) -- (graph.east -| memory-space.north);
\draw[<-] (sound.east) -- (sound.east -| memory-space.north);
\draw[->] (joystick.east) -- (joystick.east -| memory-space.north);
\draw[<->] (graph.west) -- (graph.west -| video-space.south);

\draw[<->] (memory-space) -- (ram);
\draw[<->] (video-space) -- (video-ram);

\draw[->] (graph) -- (cpu) node[midway, right] {\tiny\texttt{VBLANK}};

\end{tikzpicture}
\caption{Регистры OpenRISC 1000}\label{fig:or1k-regs}
\end{figure}

\subsection{Набор инструкций}

\todo Обрисовать.

\section{Кодовая база Simics для создания моделей процессоров}

Основа --- sample-risc, модифицированный для нужд учебного проекта:

 - убрана мноноядерность



\section{Порядок работы}

\begin{enumerate}
\item Подготовить workspace Simics
\item Собрать код sample-risc
\item \todo 

\end{enumerate}


\iftoggle{webpaper}{
    \printbibliography[title={Список литературы к занятию}]
}{}


% День 1
% 1.	[Лекция, 3 часа] Программное моделирование для задач совместной разработки аппаратуры и программ. Другие применения программных моделей. История использования. Симулятор Wind River Simics. Принципы работы, базовые понятия.
% 2.	[Практикум, 5 часов] Знакомство с Simics: создание проекта, запуск моделирования, управление и инспектирование состояния симуляции. Сборка моделей шаблонных устройств.
% День 2 
% 1.	[Лекция, 2 часа] Моделирование центральных процессоров с помощью интерпретации. Моделирование периферийных устройств с помощью симуляции дискретных событий.
% 2.	[Лекция, 1 час] Архитектура OpenRISC 1000. 
% 3.	[Практикум, 5 часов] Создание модели процессора OpenRISC 1000 с использованием фреймворка Simics Model Builder: архитектурное состояние, набор инструкций ORBIS32.
% День 3
% 1.	[Практикум, 8 часов] Продолжение реализации основной и опциональной функциональности модели OpenRISC 1000.
% а.	Набор инструкций ORBIS32.
% б.	Поддержка исключений.
% в.	Трансляция адресов: TLB и (MMU).
% г.	Периферийные устройства: Tick timer facility.
% д.	Периферийные устройства: PIC.
% е.	Модель кэша данных.
% ж.	Набор инструкций ORFPX32.
% Проверка работоспособности модели с помощью юнит-тестирования и микро-приложений, собранных с помощью GCC-or1k.
