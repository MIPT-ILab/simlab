\chapter{Моделирование платформы с архитектурой CHIP16}\label{chap:lab05}

Данная индивидуальная работа посвящена реализации модели компьютерной платформы, основанной на спецификации CHIP16~\cite{chip16-ngemu} --- полностью виртуальной системы, предназначенной для запуска простых видеоигр. Модель строится на основе API Simics и оформляется как набор модулей и сценариев для данного симулятора.

\section{Исходные спецификации CHIP16}

CHIP16 --- это RISC-подобный процессор фон-Неймана, работающий на частоте 1~МГц, имеющий 64 кбайт ОЗУ, со спрайтовой 16-цветной графикой разрешением 320×240, джойстиками с 8 кнопками и одноканальной звуковой картой.

\subsection{Существующие приложения}

\begin{itemize}
    \item Референсный симулятор --- \textsc{mash16}~\cite{chip16-mach16}.
    \item Описание устройств, набора инструкций и периферии~\cite{chip16-machspec}.
    \item Готовые образы памяти с приложениями, собранными для этой архитектуры~\cite{chip16-roms}.
\end{itemize}

\section{Структурная схема платформы}

Базовые узлы системы показаны на рис.~\ref{fig:chip16-platform}. Каждый обозначенный блок представляет собой один класс Simics, имя которого стоит после двоеточия.

\begin{figure}[htbp]
\centering
\begin{tikzpicture}[>=latex, node distance=0.5cm, font=\small]
\begin{scope}[minimum width=4cm]
    \node[draw] (cpu) {\textsc{cpu}: chip16};
    \node[draw, below = of cpu.south west, anchor=north west] (graph) {\textsc{graphics}: graph16};
    \node[draw, below = of graph.south west, anchor=north west] (sound) {\textsc{sound}: snd16};
    \node[draw, below = of sound.south west, anchor=north west] (joystick) {\textsc{joystick}: input16};
\end{scope}
    
\begin{scope}[minimum width = 3.9cm, inner xsep = 0pt]
     \node[draw, rotate=90, right = 1cm of joystick.south east, anchor=north west] (memory-space) {\textsc{cspace}: memory-space};
    \node[draw, rotate=90, right = 1cm of memory-space.south, anchor = north] (ram) {\textsc{ram}: ram};
    
    \node[draw, rotate=90, left = 1cm of cpu.north west, anchor=south east] (video-space) {\textsc{vspace}: memory-space};
    \node[draw, rotate=90, left = 1cm of video-space.south, anchor = south] (video-ram) {\textsc{video-ram}: ram};
\end{scope}

\draw[<->] (cpu.east) -- (cpu.east -| memory-space.north);
\draw[<-] (graph.east) -- (graph.east -| memory-space.north);
\draw[<-] (sound.east) -- (sound.east -| memory-space.north);
\draw[->] (joystick.east) -- (joystick.east -| memory-space.north);
\draw[<->] (graph.west) -- (graph.west -| video-space.south);

\draw[<->] (memory-space) -- (ram);
\draw[<->] (video-space) -- (video-ram);

\draw[->] (graph) -- (cpu) node[midway, right] {\tiny\texttt{VBLANK}};

\end{tikzpicture}
\caption{Схема соединения блоков}\label{fig:chip16-platform}
\end{figure}

\section{Модификации спецификации для полноплатформенной модели}

Оригинальная документация не описывает некоторые важные для практического построения детали взаимодействия узлов. Поэтому для нужд задания вводятся уточнения по принципам работы ряда узлов.

\paragraph{VBLANK.} Инструкция \textsc{vblank} введена для синхронизации ЦПУ с видеопроцессором. В данной работе её семантика изменена. \textsc{vblank} вызывает остановку процессора до поступления следующего прерывания.

\paragraph{Прерывания.} Для обеспечения работы \textsc{vblank} используется прерывание от видеокарты до процессора.

\paragraph{Инструкции DRW, PAL, SNG, CLS, BGC, SPR, FLIP, SNDx, SNP}. Данные инструкции предназначены для работы с периферией (видео и звуком). Для их реализации необходимо иметь однонаправленный канал для передачи сообщений. Для этих нужд используется  часть резервированного диапазона адресов I/O. Карта памяти выглядит следующим образом.

\todo Описать форматы пакетов аудио- и видеосообщений.

\begin{tabular}{rrl}
Диапазон адресов         & Длина & Назначение \\
\texttt{0x0000 -- 0xfdef}& 65008 & ОЗУ        \\
\texttt{0xfdf0 -- 0xffef}& 512   & Стек       \\
\texttt{0xfff0 -- 0xfff1}& 2     & Джойстик 1 \\
\texttt{0xfff2 -- 0xfff3}& 2     & Джойстик 2 \\
\texttt{0xfff4 -- 0xfff5}& 2     & Звуковая карта \\
\texttt{0xfff6 -- 0xfff9}& 4     & Видеокарта \\
\texttt{0xfffa -- 0xffff}& 6     & Зарезервировано \\
\end{tabular}


\section{Ход работы}

В данной секции разобраны общие вопросы организации разработки.
\subsection{Технология разработки}

\begin{itemize}
\item Хранение кода --- в Git \url{https://github.com/yulyugin/ilab-simics}. Лицензия кода --- закрытая, согласно договору предоставления Intel Academic SLA 1.0 (см. пункт 6.2 соглашения). 

\item Документация --- в вики \url{https://github.com/yulyugin/ilab-simics/wiki}.

\item Распределение задач --- по одному модулю Simics на одного--двух выполняющих.
\item Промежуточная проверка качества --- юнит тесты для отдельных устройств.
\item Финальный продукт работы --- набор моделей и связывающий их скрипт (построенный на основе стандартной цели cosim)
\end{itemize}

\subsection{ЦПУ}

Основа для модели --- модифицированный \texttt{sample-risc}. \todo Подготовить код.

\subsection{Видеоконтроллер}

Основа для модели --- \texttt{sample-device-c}. Отрисовка изображения производится через SDL. Пример работы с экраном и клавиатурой из SDL: \url{http://www.aaroncox.net/tutorials/2dtutorials/sdlkeyboard.html}. Для 2.0: \url{http://www.sdltutorials.com/sdl-tutorial-basics}, \url{https://wiki.libsdl.org/MigrationGuide}.

Пример на русском языке: \url{http://habrahabr.ru/post/134936/}

\subsection{Джойстик}

Основа для модели --- \texttt{sample-device-c}. Ввод с хозяйской клавиатуры производится через SDL.

\section{Минимум и максимум цели проекта}

В зависимости от полноты выполнения студентами подзадач проекта выделяются следующие вехи, обозначающие достижение следующей ступени к симуляции, по сравнению с референсной моделью \textsc{mash16}.

\begin{enumerate}
    \item Модель способна исполнять образ памяти, написанный участниками проекта.
    \item Модель способна исполнять образ памяти из директории Demo (графическое приложение без звука и ввода).
    \item Модель способна исполнять образ памяти из директории Demo (графическое приложение со звуком и без ввода).
    \item Модель способна исполнять образ памяти из директории Games (графическое приложение со звуком и вводом с джойстика).
\end{enumerate}

% \subsection{Предложения для расширения архитектуры CHIP16}
% 
% \paragraph{Таймер.} Программируемый таймер.
% 
% \paragraph{Клавиатурный ввод.}
% 
% \paragraph{Дисковый накопитель.}
% 
% \paragraph{Контроллер прерываний.}

\section{Порядок занятий и заданий}

\begin{tabular}{p{0.4\textwidth}p{0.4\textwidth}}
\textbf{Занятие} & \textbf{Задание} \\\hline
Системы контроля версий SVN (Git?) & Установить Linux, скачать репозиторий проекта, изменить файл, зафиксировать изменения в репозитории \\
Сборка проекта: модули, модели, workspace & Собрать заглушки модулей, создать недостающие устройства (распределение устройств по владельцам) \\
Структура кода моделей Simics: атрибуты, интерфейсы, функция init_local & Начать заполнять устройства архитектурным состоянием \\
Архитектура CHIP16 & Продолжать работу над устройствами \\
Моделирование ЦПУ через интерпретацию; моделирование графики & Реализовать первые инструкции в процессоре; подключить библиотеку SDL к сборке проекта\\
Моделирование ввода с клавиатуры. Интерактивность симуляции. MMIO. Прерывания & Продолжать писать модели устройств \\
Тестирование устройств. Среда Simics для конфигурации соединения устройств симуляции. & Запустить скрипт targets/chip16; написать свои скрипты для отдельных устройств \\



\end{tabular}


\iftoggle{webpaper}{
    \printbibliography[title={Список литературы к занятию}]
}{}