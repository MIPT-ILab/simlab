\chapter{Язык сценариев Simics}\label{chap:scripting}

\section{Цель занятия}

В данной лабораторной работе продолжается изучение принципов работы симулятора Simics. В ней будут даны основы языка сценариев конфигурации Simics.

\section[Файлы, интерактивный ввод и опции]{Файлы сценариев, интерактивный ввод и опции командной строки}

\todo

\section{Ход работы}

\todo 

% \section{}

\section{Последовательность конструирования модели системы}

При создании симуляции она и все участвующие в ней объекты проходят две фазы, чётко отделённые друг от друга во времени, тогда как внутри каждой из них порядок инициализации компонент неопределён.

\begin{enumerate*}

\item \textit{Объявление} устройств и соединение их с помощью инициализации их атрибутов. На этом этапе ещё не произодится проверок на соответствие типов, наличие всех обязательных атрибутов. Устройства представлены так называемыми предобъектами (\abbr preobj).

\item \textit{Инстанциирование} устройств с помощью команды \texttt{instantiate-components}. На этом этапе производятся все проверки на корректность атрибутов, наличие необходимых интерфейсов. Если не найдено ошибок, то предобъекты преобразуются в полноценные объекты Simics, которые могут участввать в симуляции.

\end{enumerate*}

При необходимости добавить новые объекты в процессе симуляции описанные фазы могут быть повторены.


\section{Контрольные вопросы}

\begin{enumerate*}
\item 
\item 
\item 
\end{enumerate*}

\iftoggle{webpaper}{
    \printbibliography[title={Список литературы к занятию}]
}{}




