% Compile with XeLaTeX, TeXLive 2013 or more recent
\documentclass{beamer}

% Base packages
\usepackage{fontspec}
\usepackage{xunicode}
\usepackage{xltxtra}

\usepackage{amsfonts}
\usepackage{amsmath}
\usepackage{longtable}
\usepackage{csquotes}
\usepackage{standalone}

% Setup fonts
\newfontfamily\russianfont{CMU Serif}
\setromanfont{CMU Serif}
\setsansfont{CMU Sans Serif}
\setmonofont{CMU Typewriter Text}

% Setup Russian hyphenation. NOTE: this declaration *must* come after fontspec's font declarations,
% or a mysterious (but harmless in other respects) error "Improper `at' size (0.0pt), replaced by 10pt." would appear.
\usepackage{polyglossia}
\defaultfontfeatures{Scale=MatchLowercase, Mapping=tex-text}

\setdefaultlanguage[spelling=modern]{russian} % for polyglossia
\setotherlanguage{english} % for polyglossia

% Vector drawings 
\usepackage{tikz}
\usetikzlibrary{shapes, calc, arrows, fit, positioning, decorations, patterns, decorations.pathreplacing, chains, snakes}

% Be able to insert hyperlinks
\usepackage{hyperref}
\hypersetup{colorlinks=true, linkcolor=black, filecolor=black, citecolor=black, urlcolor=blue , pdfauthor=Grigory Rechistov <grigory.rechistov@phystech.edu>, pdftitle=Моделирование OpenRISC 1000 на Wind River Simics}
% \usepackage{url}

% Misc optional packages
\usepackage{underscore}
\usepackage{amsthm}

\usepackage{ulem} % for strikethrough
\usepackage{bytefield} % for bitefield

% A new command to mark not done places
\newcommand{\todo}[1][Напиши меня]{{\color{red}TODO\ #1}}


\title{Моделирование OpenRISC 1000}
% \subtitle{Курс «Программное моделирование вычислительных систем»}
\subject{Лекция}
\author[Григорий Речистов]{Григорий Речистов \\ \small{\href{mailto:grigory.rechistov@intel.com}{grigory.rechistov@intel.com}}}
\date{26-28 августа 2014 г.}
\pgfdeclareimage[height=0.5cm]{intel-logo}{../images/intel.png}
\logo{\pgfuseimage{intel-logo}}

\typeout{Copyright 2014 Grigory Rechistov}

\usetheme{Berlin}
\setbeamertemplate{navigation symbols}{}%remove navigation symbols

\begin{document}

\begin{frame}
\titlepage
\end{frame}

\begin{frame}
\tableofcontents
\end{frame} 


\section{Обзор \texttt{or1k}}

\begin{frame}{Что такое OpenRISC 1000}

\begin{itemize}
	\item Модульная архитектура цифровой системы с поддержкой сообщества OpenCores.
	\item Спецификация центрального процессора/SoC.
	\item Реализована в симуляторах, VHDL, FPGA и кремнии.
	\item Поддерживается Linux, Glibc и др.
\end{itemize}

\end{frame}

\begin{frame}{Почему OpenRISC 1000 для симуляции}
\begin{itemize}\pause
\item\only<2>{MIPS всем надоел.}\only<3->{\sout{MIPS всем надоел.}}
\item<4-> Модульная структура — много необязательных частей.
\item<4-> Простая архитектура.
\item<5-> Но не \textit{слишком} простая.
\end{itemize}
\end{frame}

\begin{frame}{ISA}
\centering
\begin{tikzpicture}[font=\small, node distance = 0.5cm]
    \node[draw, ellipse] (ORBIS32) {ORBIS32};
    \node[draw, ellipse, right = of ORBIS32] (ORBIS64) {ORBIS64};
    \node[draw, ellipse, below = of ORBIS32] (ORFPX32) {ORFPX32};
    \node[draw, ellipse, below = of ORBIS64] (ORFPX64) {ORFPX64};
    \node[draw, ellipse, below = of ORFPX64] (ORVDX64) {ORVDX64};
    \node[draw, ellipse, fit = (ORBIS32) (ORBIS64) (ORFPX32) (ORFPX64) (ORVDX64)] {};
\end{tikzpicture}

\end{frame}

\begin{frame}{Регистры}
\bytefieldsetup{bitwidth=0.07cm, endianness = big}
\begin{columns}
	\begin{column}{0.5\textwidth}
		\begin{bytefield}[]{32}
			\bitheader{31, 0} \\
			\bitbox{32}{PC} \\
		\end{bytefield}

		\begin{bytefield}[]{64}
			\bitheader{63, 0} \\
			\bitbox{64}{GPR0} \\
			\bitbox{64}{\dots} \\
			\bitbox{64}{GPR31} \\
		\end{bytefield}
	\end{column}

	\begin{column}{0.5\textwidth}
		\begin{bytefield}[]{32}
			\bitheader{31, 0} \\
			\bitbox{32}{VR} \\
			\bitbox{32}{UPR} \\
			\bitbox{32}{CPUCFGR} \\
			\bitbox{32}{\dots} \\
			\bitbox{32}{ESR15} \\
		\end{bytefield}
	\end{column}
\end{columns}

\end{frame}

\begin{frame}{Что ещё есть}

\begin{itemize}
\item Исключения.
\item TLB
\item MMU.
\item Cache.
\item PIC.
\item Tick timer.
\item PCU, Debug, PMU \dots
\end{itemize}

\end{frame}

% \begin{frame}{}

% \end{frame}

\section{Ход работы}

\begin{frame}[fragile]{Получение исходников модели}
\begin{verbatim}
$ git clone user@host:/path wrkspc
Cloning into 'wrkspc'...
done.
\end{verbatim}
\end{frame}

\begin{frame}[fragile]{Создание workspace}

\begin{footnotesize}
\begin{verbatim}
$ /opt/simics/simics-4.6/simics-4.6.112/bin/workspace-setup wrkspc
Workspace created successfully
$ cd wrkspc
\end{verbatim}
\end{footnotesize}

\end{frame}

\begin{frame}{Сборка}

\texttt{\$ make}

\end{frame}

\begin{frame}[fragile]{Юнит-тест}

\begin{verbatim}
$ ./bin/test-runner
...
Ran 3 tests in 3 suites in 2.171517 seconds.
All tests completed successfully.
\end{verbatim}

\end{frame}

\begin{frame}[fragile]{Структура кода}
\begin{tiny}
\begin{verbatim}
$ ls -1 modules/or1k/
Makefile
or1k-cycle.c
or1k-cycle.h
or1k-exec.c
or1k-exec.h
or1k-frequency.c
or1k-frequency.h
or1k-memory.c
or1k-memory.h
or1k-queue.c
or1k-queue.h
or1k-step.c
or1k-step.h
or1k.c
or1k.h
commands.py
event-queue-types.h
event-queue.c
event-queue.h
\end{verbatim}
\end{tiny}

\end{frame}

\begin{frame}[fragile]{Атрибуты}

\begin{tiny}
\begin{verbatim}
simics> api-help SIM_register_typed_attribute
Help on API keyword "SIM_register_typed_attribute":

// defined in simics/base/conf-object.h
#include <simics/device-api.h>    // in C/C++
// always available in DML

int SIM_register_typed_attribute(conf_class_t * NOTNULL cls, const char * NOTNULL name, 
	get_attr_t get_attr, lang_void *user_data_get, set_attr_t set_attr, lang_void
    *user_data_set, attr_attr_t attr, const char *type, const char *idx_type, const char *desc);

// available in Python
\end{verbatim}
\end{tiny}

\end{frame}

\begin{frame}[fragile]{Интерфейсы}

\begin{tiny}
\begin{verbatim}
simics> api-help SIM_register_interface
Help on API keyword "SIM_register_interface":

// defined in simics/base/conf-object.h
#include <simics/device-api.h>    // in C/C++
// always available in DML

int SIM_register_interface(conf_class_t * NOTNULL cls, const char * NOTNULL name, 
const interface_t * NOTNULL iface);

// available in Python
\end{verbatim}
\end{tiny}
\end{frame}

\begin{frame}{Задание}

\begin{enumerate}
\item Выбрать часть функциональности, требующую реализации.
\item Спланировать процесс реализации.
\item Реализовать в коде.
\item Добиться успешной сборки.
\item Написать юнит-тест.
\item Добиться успешного завершения юнит-теста.
\item GOTO 1.
\end{enumerate}

\end{frame}


\section{Литература}

\begin{frame}[allowframebreaks]{Литература}
\begin{thebibliography}{99}
    \bibitem{simbook} Основы программного моделирования ЭВМ. Учебное пособие / Г. Речистов, А. Иванов, П. Шишпор, Н. Щелкунов, Д. Гаврилов, В. Пентковский. — Издательство МФТИ, дек. 2012. — ISBN 978-5-7417-0469-1
	\bibitem{or1k-isa-spec}  Open Cores OpenRISC 1000 Architecture Manual. Architecture Version 1.0 Document Revision 0. — 2012. — URL: \url{http://opencores.org/websvn,filedetails?repname=openrisc&path=/openrisc/trunk/docs/openrisc-arch-1.0-rev0.pdf}
	\bibitem{proc-integration} Processor Model Integration Guide 4.6 / Wind River. — 2014.
	\bibitem{model-builder} Simics Model Builder User Guide 4.6 / Wind River. — 2014.
	
\end{thebibliography}
\end{frame}


\section{Конец}
% The final "thank you" frame 
\begin{frame}

{\huge{Спасибо за внимание!}\par}

\vfill

Слайды и материалы курса доступны по адресу \url{http://bit.ly/1y1lZF1} % http://atakua.doesntexist.org/wordpress/tag/or1k/

\vfill

\tiny{\textit{Замечание}: все торговые марки и логотипы, использованные в данном материале, являются собственностью их владельцев. Представленная точка зрения отражает личное мнение автора.
%Материалы доступны по лицензии Creative Commons Attribution-ShareAlike (Атрибуция — С сохранением условий) 4.0 весь мир (в т.ч. Россия и др.). Чтобы ознакомиться с экземпляром этой лицензии, посетите \url{http://creativecommons.org/licenses/by-sa/4.0/}
}

\end{frame}

% \section{Резерв}

\end{document}
