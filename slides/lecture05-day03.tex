% Compile with XeLaTeX, TeXLive 2013 or more recent
\documentclass{beamer}

% Base packages
\usepackage{fontspec}
\usepackage{xunicode}
\usepackage{xltxtra}

\usepackage{amsfonts}
\usepackage{amsmath}
\usepackage{longtable}
\usepackage{csquotes}
\usepackage{standalone}

% Setup fonts
\newfontfamily\russianfont{CMU Serif}
\setromanfont{CMU Serif}
\setsansfont{CMU Sans Serif}
\setmonofont{CMU Typewriter Text}

% Setup Russian hyphenation. NOTE: this declaration *must* come after fontspec's font declarations,
% or a mysterious (but harmless in other respects) error "Improper `at' size (0.0pt), replaced by 10pt." would appear.
\usepackage{polyglossia}
\defaultfontfeatures{Scale=MatchLowercase, Mapping=tex-text}

\setdefaultlanguage[spelling=modern]{russian} % for polyglossia
\setotherlanguage{english} % for polyglossia

% Vector drawings 
\usepackage{tikz}
\usetikzlibrary{shapes, calc, arrows, fit, positioning, decorations, patterns, decorations.pathreplacing, chains, snakes}

% Be able to insert hyperlinks
\usepackage{hyperref}
\hypersetup{colorlinks=true, linkcolor=black, filecolor=black, citecolor=black, urlcolor=blue , pdfauthor=Grigory Rechistov <grigory.rechistov@phystech.edu>, pdftitle=Моделирование OpenRISC 1000 на Wind River Simics}
% \usepackage{url}

% Misc optional packages
\usepackage{underscore}
\usepackage{amsthm}

\usepackage{ulem} % for strikethrough
\usepackage{bytefield} % for bitefield

% A new command to mark not done places
\newcommand{\todo}[1][Напиши меня]{{\color{red}TODO\ #1}}


\title{Моделирование периферийных устройств}
% \subtitle{Курс «Программное моделирование вычислительных систем»}
\subject{Лекция}
\author[Григорий Речистов]{Григорий Речистов \\ \small{\href{mailto:grigory.rechistov@intel.com}{grigory.rechistov@intel.com}}}
\date{26-28 августа 2014 г.}
\pgfdeclareimage[height=0.5cm]{intel-logo}{../images/intel.png}
\logo{\pgfuseimage{intel-logo}}

\typeout{Copyright 2014 Grigory Rechistov}

\usetheme{Berlin}
\setbeamertemplate{navigation symbols}{}%remove navigation symbols

\begin{document}

\begin{frame}
\titlepage
\end{frame}

\begin{frame}
\tableofcontents
\end{frame} 

\section{«Полная» платформа}

\begin{frame}{Что у нас есть}
\centering 

\begin{tikzpicture}[>=stealth, font=\small, node distance = 0.4cm]
\begin{scope}[minimum height=0.8cm]
	\node[draw, ] (cpu) {CPU};
	
	% \node[draw, right=3cm of cpu, ] (pic) {PIC};

	\node[draw, below=of cpu, text width=3cm, align = center, ] (dram) {DRAM};
\end{scope}

\draw[<->] (cpu) -- (dram);

\end{tikzpicture}

\end{frame}

\begin{frame}{Что хочется}
\centering 

\begin{tikzpicture}[>=stealth, font=\small, node distance = 0.4cm]

\begin{scope}[minimum height=0.8cm]
	\node[draw, ] (cpu) {CPU};
	\node[draw, below=of cpu] (mmu) {MMU};
	\node[draw, right=2cm of cpu, ] (pic) {PIC};

	\node[draw, below=of mmu, text width=3cm, align = center, ] (dram) {DRAM};
	\node[draw, right=of pic, ] (pit) {Tick};

\end{scope}

\draw[<->] (cpu) -- (mmu);
\draw[<->] (mmu) -- (dram);

\draw[<->] (mmu) -| (pit);

\draw[->, dashed] (pic) -- (cpu);
\draw[->, dashed] (pit) -- (pic);

\end{tikzpicture}


\end{frame}

\section{Периферийные устройства}

\begin{frame}{Устройства}

\begin{itemize}
\item Пишутся на DML или на C.
\item \texttt{tick} — на DML.
\item \texttt{joy16} — на C.
\end{itemize}

\end{frame}


\begin{frame}{Что ещё потребуется}

\begin{itemize}
\item Обработка прерываний в ЦПУ
\item Код, реагирующий на прерывания.
\item Соединение всех устройств в платформу.
\end{itemize}

\end{frame}



\section{Литература}

\begin{frame}[allowframebreaks]{Литература}
\begin{thebibliography}{99}
	\bibitem{or1k-isa-spec}  Open Cores OpenRISC 1000 Architecture Manual. Architecture Version 1.0 Document Revision 0. — 2012. — URL: \url{http://opencores.org/websvn,filedetails?repname=openrisc&path=/openrisc/trunk/docs/openrisc-arch-1.0-rev0.pdf}
	\bibitem{model-builder} Simics Model Builder User Guide 4.6 / Wind River. — 2014.
	\bibitem{dml} Simics DML Reference Guide 4.6 / Wind River. — 2014.
	
\end{thebibliography}
\end{frame}


\section{Конец}
% The final "thank you" frame 
\begin{frame}

{\huge{Спасибо за внимание!}\par}

\vfill

Слайды и материалы курса доступны по адресу \url{http://bit.ly/1y1lZF1} % http://atakua.doesntexist.org/wordpress/tag/or1k/

\vfill

\tiny{\textit{Замечание}: все торговые марки и логотипы, использованные в данном материале, являются собственностью их владельцев. Представленная точка зрения отражает личное мнение автора.
%Материалы доступны по лицензии Creative Commons Attribution-ShareAlike (Атрибуция — С сохранением условий) 4.0 весь мир (в т.ч. Россия и др.). Чтобы ознакомиться с экземпляром этой лицензии, посетите \url{http://creativecommons.org/licenses/by-sa/4.0/}
}

\end{frame}

% \section{Резерв}

\end{document}
