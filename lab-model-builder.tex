\chapter{Создание программных моделей}\label{chap:lab-model-builder}

\section{Цель занятия}

В данной лабораторной работе продолжается изучение принципов работы симулятора Simics. В ней будут определены основные понятия, используемые при работе с симуляцией и при написании моделей устройств. В качестве основного языка программирования моделей будет использоваться Си.

Кроме использования готовых моделей устройств из поставляемых в составе Simics, имеется возможность создавать свои собственные модели различных устройств. Для этого симулятор предоставляет среду сборки, API для взаимодействия с симулятором и специальный язык DML для быстрой разработки периферийных устройств\footnote{В данной работе DML не будет использоваться; однако Simics API доступен из всех поддерживаемых языков, включая Python, Си и C++.}. Подробное описание принципов написания моделей периферийных устройств дано в~\cite{model-builder}; особенности интеграции моделей центральных процессоров описываются в~\cite{processor-integration}.

\section{Модули Simics}
Базовая единица загрузки нового кода Simics --- \textit{модуль}, разделяемая библиотека (в Linux это файлы с расширением .so или .pymod), использующая API Simics. Каждый класс объектов должен содержаться максимум в одном модуле. При этом один модуль может содержать в себе более одного класса моделей.

Для просмотра текущих загруженных в симуляцию модулей используйте команду \texttt{list-modules -l}. Например, в <<пустой>> симуляции этот список выглядит так (флаг \texttt{-v} позволяет увидеть больше информации о том, откуда загружены модули):

\begin{lstlisting}
simics> list-modules -v -l
Name                    Status   ABI  Build ID  Thread-safe
Path
-----------------------------------------------------------
breakpoint-manager      Loaded  4052      4145  Yes
/opt/simics/simics-4.6/simics-4.6.103/linux64/lib/breakpoint-manager.so
software-tracker        Loaded  4052      4145  Yes
/opt/simics/simics-4.6/simics-4.6.103/linux64/lib/software-tracker.pymod
software-tracker-iface  Loaded  4052      4145  Yes
/opt/simics/simics-4.6/simics-4.6.103/linux64/lib/software-tracker-iface.so
\end{lstlisting}

Поиск стандартных модулей, доступных для загрузки, проводится при старте Simics в директориях инсталляции базового и других пакетов. Пользовательские пакеты ищутся в текущем workspace в подпапке \texttt{linux64/lib}. Они помещаются туда в результате успешной компиляции из исходных кодов.

Для загрузки некоторого модуля из командной строки вручную используется команда \texttt{load-module <name>}. После этого становятся доступны все классы и команды, определённые в нём, т.е. можно создавать новые объекты для моделей устройств новых классов.

\section[Исходный код и сборка]{Исходный код и сборка моделей устройств}

Код всех модулей дожен быть размещён в текущем workspace, в поддиректории \texttt{modules}. По общим соглашениям имя директории должно совпадать с именем модуля.
Система сбоки модулей Simics довольно сложна и требует строгого соблюдения процедуры. Она основана на GNU Make~\cite{gmake} и использует компилятор GCC на всех поддерживаемых хозяйских платформах.

\subsection{Генерация шаблонного устройства}

Для начала необходимо сгенерировать <<скелет>> нового устройства --- минимально необходимый набор файлов. Это позволит начать работать с уже компилирующимся примером.

Из workspace выполните команду:

\begin{lstlisting}
$ ./bin/workspace-setup --copy-device sample-device-c
\end{lstlisting}

Теперь в директории \texttt{modules} появился новый элемент \texttt{sample-device-c} с исходными файлами:
\begin{lstlisting}
$ ls -R modules/
modules/:
sample-device-c

modules/sample-device-c:
Makefile  commands.py  sample-device.c  test

modules/sample-device-c/test:
SUITEINFO  s-sample-c.py
\end{lstlisting}

\subsection{Сборка с помощью Make}

Для запуска сборки достаточно использовать программу \texttt{make} с именем цели, совпадающей с именем модуля:
\begin{lstlisting}
make sample-device-c
=== Environment Check ===
'/home/simics/workspace' is up-to date
gcc version 4.6.2
=== Building module "sample-device-c" ===
        module_id.c
DEP     module_id.d
DEP     sample-device.d
CC      sample-device.o
CC      module_id.o
CCLD    sample-device-c.so
        mod_sample_device_c_commands.pyc
\end{lstlisting}

Далее мы будем работать с файлами внутри директории \texttt{sample-device-c}.

\subsection{Структура Makefile}

В целом синтаксис языка Makefile позволяет писать достаточно сложные и запутанные правила для сборки приложений. В Simics используется собственная система соглашений для упрощения определения модулей. В простейшем случае достаточно перечислить все требуемые файлы с исходными кодами в переменной \texttt{SRC_FILES}. Кроме того, имя нового класса должно указываться в переменной \texttt{MODULE_CLASSES} в \texttt{Makefile}.

\section{Регистрация нового класса}
Перейдём теперь к деталям того, как использовать Simics API для создания нового класса устройств.
При загрузке любого модуля, написанного на Си, Simics исполняет из него единственную функцию с именем \texttt{init_local()}. В ней должна быть выполнена регистрация нового класса объектов с указанием всех его свойств.

Для этого используется функция \texttt{SIM_register_class()}. В примере \texttt{sample-device-c}:

\begin{lstlisting}
conf_class_t *class = SIM_register_class("sample-device-c", &funcs);
\end{lstlisting}

Первый аргумент этой функции --- имя класса, второй --- структура, задающая функцию-конструктор новых копий объекта, а также строки с описанием.

В результате успешного завершения возвращается ссылка на \texttt{class}. Только что созданный класс пока что почти пуст --- он не экспортирует наружу никаких свойств моделируемых объектов. Перейдём к добавлению двух важнейших аспектов любого полноценного класса --- регистрации атрибутов и интерфейсов.

\section{Атрибуты}

Атрибуты позволяют видеть и изменять состояние объекта из командной строки Simics. Кроме того, они необходимы для того, чтобы механизм точек сохранения мог корректно записывать состояние на диск и впоследствии загружать его.

Структура объектов \texttt{sample-device-c} задана следующей структурой:
\begin{lstlisting}
typedef struct {
        /* Simics configuration object */
        conf_object_t obj;

        /* device specific data */
        unsigned value;

} sample_device_t;
\end{lstlisting}

Здесь \texttt{conf_object_t obj} --- <<базовый класс>> иерархии объектов Simics, он должен присутствовать в любом объекте. Собственно устройство имеет только один регистр, названный \texttt{value}. Для того, чтобы иметь возможность читать и писать его значение, с этим полем ассоциируется атрибут, также названный value:

\begin{lstlisting}
SIM_register_typed_attribute(
        class, "value",
        get_value_attribute, NULL, set_value_attribute, NULL,
        Sim_Attr_Optional, "i", NULL,
        "The <i>value</i> field.");
\end{lstlisting}

Здесь \texttt{class} --- ранее созданный класс устройства, \texttt{get_value_attribute()} и \texttt{set_value_attribute()} --- функции для чтения и записи\footnote{Т.н. getter и setter.} значения, \texttt{"i"} --- тип атрибута (в данном случае это целое число), строка \texttt{"The <i>value</i> field."} --- справка о назначении атрибута. Необязательные аргументы функции \texttt{SIM_register_typed_attribute()} равны \texttt{NULL}.

Особое внимание следует обратить на то, как устроены getter и setter. Значения атрибутов, которые могут иметь довольно сложные типы, упакованы в специальный класс \texttt{attr_value_t}. Для получения значений используется семейство функций \texttt{SIM_attr_*}.

\section{Интерфейсы}

Назначение методов, группируемых в интерфейсы, состоит в изменении состояния устройств, а также чтения значений. В отличе от атрибутов, способных делать то же самое, интерфейсы служат для представления архитектурных возможностей устройств. Атрибуты не имеют выражения в реальной аппаратуре, тогда как интерфейсы напрямую отображаются на шины, сигналы, протоколы и т.п.

Каждый интерфейс имеет уникальное имя и фиксированный набор методов, объединённых общей целью. Модель, желающая предоставлять некоторый интерфейс для других устройств, обязана реализовать один или несколько его методов и затем объявить его доступным. Устройство, имеющее ссылку на объект, может получить по нему заявленные интерфейсы и вызывать включённые в него методы. Таким образом, в Simics интерфейсы предоставляют объектно-ориентированную парадигму для взаимодействия отдельных моделей.

Например, один из атрибутов процессора, настраиваемый на этапе инициализации модели, --- это \texttt{physical_memory}, его тип \texttt{o}, т.е. <<объект>>. Допустим, что \texttt{cpu->physical_memory = mem}. По указателю на объект \texttt{mem} процессор может извлечь из него реализацию интерфейса \texttt{memory-space}, который содержит методы \texttt{read}, \texttt{write}, \texttt{access} и др. для работы с пространствами памяти.

В случае \texttt{sample-device-t} регистрируются два интерфейса --- \texttt{sample_interface}  и \texttt{io_memory}.
Пример регистрации первого из них:
\begin{lstlisting}
static const sample_interface_t sample_iface = {
        .simple_method = simple_method
};
SIM_register_interface(class, SAMPLE_INTERFACE, &sample_iface);
\end{lstlisting}

Здесь \texttt{sample_iface} --- структура из указателей на функции, которая передаётся функции \texttt{SIM_register_interface()}. \texttt{class} --- тот же самый класс, что был получен при регистрации класса, \texttt{SAMPLE_INTERFACE} --- строка\footnote{По соглашениям строки с названиями интерфейсов хранятся в \#define'ах, названных из заглавных букв имени, т.е. \texttt{SAMPLE_INTERFACE} эквивалентно \texttt{"sample_interface"}.} с именем интерфейса.

\section{Ход работы}

\todo
% \section{}

\section{Контрольные вопросы}

\begin{enumerate*}
\item Чем отличается вывод команды \texttt{list-modules} с флагом \texttt{-l} и без него?
\item Некоторые найденные модули по той или иной причине могут быть отвергнуты при загрузке. Увидеть их список можно с помощью команды \texttt{list-failed-modules}. Загрузите модель Viper и определите причины, по которым список, выдаваемый командой \texttt{list-failed-modules}, не пуст.

\item По каким причинам не следует использовать манипуляцию атрибутами одного устройства из другого?

\item Объясните, для чего служит атрибут \texttt{add_log}, регистрируемый в конце \texttt{init_local()}.
\end{enumerate*}

\iftoggle{webpaper}{
    \printbibliography[title={Список литературы к занятию}]
}{}

